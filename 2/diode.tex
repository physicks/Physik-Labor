%        File: template.tex
%     Created: Don Nov 25 12:00  2010 C
% Last Change: Don Nov 25 12:00  2010 C
%
\documentclass[a4paper]{article}
\usepackage[]{amsmath}
\usepackage{ngerman}
\usepackage[latin1]{inputenc}
\usepackage[]{graphicx}
\usepackage{fancyhdr}
\usepackage{epstopdf}

\pagestyle{fancy}
\fancyhead{}
\fancyfoot{}
\fancyhead[L]{Armin F�rst, Michael Pusterhofer}
\fancyhead[C]{Diode}
\fancyhead[R]{27}
\fancyfoot[C]{\thepage/10}

\begin{document}
\section{Aufgabenstellung}
\begin{itemize}
  \item Aufnehmen der Strom- Spannungskennlinie einer Halbleiterdiode.
  \item	Gleichrichtersdchaltung mit dieser Diode aufbauen. (mit 2 verschiedenen Kondensatoren)
  \item	Aufnehmen der Spannungsverl�ufe an 3 charakteristischen Punkten.
  \item Berechnen des maximalen Aufladestrom am Kondensator in den 2 Schaltungen mit Hilfe des maximalen Spannungswertes am Shunt.
\end{itemize}

\section{Voraussetzungen}
In Halbleiterdioden werden die Eigenschaften eines sogenannten p-n-�berganges
genutzt.

Die Diode wird mit dem Plus-Pol der angelegten Spannung an der p-Schicht und dem Minuspol
an der n-Schicht in Durchlassrichtung betrieben.
In Durchlassrichtung gibt es eine Schwellspannung (Diffusionsspannung), ab der der Durchlassstrom eine
nennenswerte Gr��e (ca. 1/10 des maximal erlaubten Stromes) annimmt. Sie ist vom verwendeten
Basismaterial abh�ngig und betr�gt im Fall von Silizium ca. 0.7 V.

Durch Kombination einer oder mehrerer Dioden mit anderen Bauelementen wie
Widerstand und Kondensator lassen sich Schaltkreise zur Umwandlung einer
Wechselspannung in eine Gleichspannung, genannt Gleichrichterschaltung, aufbauen.

\section{Versuchsaufbau}
	\subsection{Aufbau zur Aufnahme der Diodenkennlinie}
	\begin{figure}[h]
			\centering
			\includegraphics{aufbau_diode.jpg}
			\caption{Aufbau zur Messung der Diodenkennlinie. Die Widerst�nde betragen zusammen ca 56$\Omega$, und wurden verwenden da kein geeigneter 50$\Omega$ Widerstand vorhanden war.}
			\label{fig:aufbau_diode}
	\end{figure}
	\clearpage
	\subsection{Einweggleichrichter}
	\begin{figure}[h]
			\centering
			\includegraphics{aufbau_einweg.jpg}
			\caption{Aufbau zur Aufnahme des Strom- Spannungsverlaufs bei einem Einweggleichrichter. Die 3 Punkte markieren die Messpunkte. Die Messungen wurden f�r einen Kondensator mit 1000$\mu$ und einen mit 				100$\mu$ durchgef�hrt. $R_M$ ist ein Shunt zur Strommessung mit 1$\Omega$. $R_L$ = Lastwiderstand mit 100$\Omega$.}
			\label{fig:aufbau_einweg}
	\end{figure}

\section{Ger�teliste}
\begin{table}[ht]
		\caption{Ger�teliste}
		\centering
		\begin{tabular}{c|c}
			Ger�t & Inventarnummer \\\hline \hline
			Diode IIV 5408 & \\
			Shunt 1$\Omega$ & \\
			2x 22$\Omega$ Widerstand & \\
			12$\Omega$ Widerstand & \\
			1mF Kondensator & \\
			100$\mu$F Kondensator & \\
			Widerstandskaskade RD-1000 & \\
			Triple Power Suply HM 8040-3 & \\
			Funktionsgenerator PHYWE & \\
			Oszilloskop DSO-220-USB & 3A-Pr.\\
			2x Fluke 175 & \\
			PC mit DSO-220-USB Programm & \\
		\end{tabular}
	\label{tab:list_tools}
\end{table}

\clearpage

\section{Messung}
	\subsection{Diodenkennlinie}
		\begin{table}[ht]
				\caption{Messung der Diodenkennlinie in Durchlassrichtung}
				\begin{itemize}
					\item N \dots  Messnummer
					\item U \dots  Spannung an der Diode, $\Delta$U = $\pm$ (1,0\% + 3Digits)
					\item I \dots  Strom, $\Delta$I = $\pm$ (1,5\% + 3Digits)
				\end{itemize}
				\centering
				\begin{tabular}{c|c|c}
					N & U/V & I/mA \\\hline \hline
				 1 & 0.000 & -0.01 \\\hline
				 2 & 0.490 & -0.01 \\\hline
				 3 & 0.100 & -0.01 \\\hline
				 4 & 0.149 & -0.02 \\\hline
				 5 & 0.195 & -0.01 \\\hline
				 6 & 0.246 & -0.01 \\\hline
				 7 & 0.306 & -0.01 \\\hline
				 8 & 0.344 & -0.01 \\\hline
				 9 & 0.393 & 0 \\\hline
				 10 & 0.445 & 0.05 \\\hline
				 11 & 0.498 & 0.26 \\\hline
				 12 & 0.548 & 0.92 \\\hline
				 13 & 0.600 & 3.29 \\\hline
				 14 & 0.648 & 9.87 \\\hline
				 15 & 0.684 & 24.07 \\\hline
				 16 & 0.701 & 37.16 \\\hline
				 17 & 0.725 & 70.5 \\\hline
				 18 & 0.749 & 133.1
				\end{tabular}
			\label{tab:list_diode}
		\end{table}

	\subsection{Einweggleichrichter}
		\subsubsection{Spannungs- und Stromverl�ufe mit 1000$\mu$ F Kondesator}
		F�r den Spannungsverlauf am Lastwiederstand siehe Abbildung 4 auf Seite 6. \linebreak
		F�r den Stromverlauf am Kondensator siehe Abbildung 5 auf Seite 7.
		\subsubsection{Spannungs- und Stromverl�ufe mit 100$\mu$ F Kondesator}
		F�r den Spannungsverlauf am Lastwiderstand siehe Abbildung 6 auf Seite 8. \linebreak
		F�r den Stromverlauf am Kondensator siehe Abbildung 7 auf Seite 9.
		\subsubsection{Strom- und Spannungsverl�ufe ohne Kondesator}
		Siehe Abbildung 8 auf Seite 10.
\clearpage
\section{Rechnung/Auswertung}
	\subsection{Diodenkennlinie}
		\begin{figure}[ht]
			\centering
			\includegraphics{diodenkennlinie.eps}
			\caption{Diodenkennlinie in Durchlassrichtung. Strom in mA �ber Spannung in Volt.}
			\label{fig:diodenkennlinie}
		\end{figure}
		
	\subsection{Maximaler Strom}
	Da ein 1$\Omega$ Widerstand verwenden wurde kann der Maximale Strom direkt aus den entsprechenden Abbildungen abgelesen werden
		\subsubsection{Maximaler Strom mit 1mF Kondensator}
			Aus Abbildung 5 auf Seite 7: I = (600 $\pm$ 100)mA
		\subsubsection{Maximaler Strom mit 100$\mu$F Kondensator}
			Aus Abbildung 7 auf Seite 9: I = (240 $\pm$ 40)mA

\clearpage
\section{Zusammenfassung}
	\subsection{Diodenkennlinie}
	Die aufgenommene Diodenkennlinie auf Abbildung 3 Seite 4 sieht wie eine typische Diodenkennlinie einer Siliziumdiode aus.
	\subsection{Einweggleichrichter}
	Man erkennt an den Spannungsverl�ufen am Lastwiderstand sehr gut, dass man eine bessere Gl�ttung durch einen gr��eren Kondensator erreicht.
	Ein gr��erer Kondensator verursacht aber auch einen gr��eren Ladestrom, was man beim Schaltungsbau in betracht ziehen muss um nicht die Diode zu zerst�ren.
	\subsection{Maximaler Ladestrom mit 1mF Kondensator}
	I = (600 $\pm$ 100)mA
	\subsection{Maximaler Ladestrom mit 100$\mu$F Kondensator}
	I = (240 $\pm$ 40)mA
\end{document}


