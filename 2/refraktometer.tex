%     Created: Don Nov 25 12:00  2010 C
% Last Change: Don Nov 25 12:00  2010 C
%
\documentclass[a4paper]{article}
\usepackage[]{amsmath}
\usepackage{ngerman}
\usepackage[utf8]{inputenc}
\usepackage[]{graphicx}
\usepackage{fancyhdr}
\usepackage{lastpage}

\pagestyle{fancy}
\fancyhead{}
\fancyfoot{}
\fancyhead[L]{Michael Pusterhofer}
\fancyhead[C]{Refraktometer}
\fancyhead[R]{G27}
\fancyfoot[C]{\thepage/\pageref{LastPage} }

\begin{document}
\section{Aufgabenstellung}
\begin{itemize}
  \item Bestimmen der Brechzahl einer Flüssigkeit mit dem Abbe-Refraktometer bei der Wellenlänge einer Natrium D-Linie.
  \item Überprüfen des Ergebnisses mit einem Zeiss Refraktometer.
\end{itemize}

\section{Voraussetzungen}
  \subsection{Brechungsgesetz}
\begin{align}
  \frac{n_1}{n_2}=\frac{sin(\gamma_2)}{sin(\gamma_1)}
  \label{bechungsgesetz}
\end{align}
  \subsection{Definition der Brechzahl}
  \begin{align}
    n_m = \frac{c_0}{c_m}=\frac{sin(\gamma_0)}{sin(\gamma_m)}
    \label{brechzahl}
  \end{align}
  \subsection{Dispersion}
  Dispersion tritt auf wenn eine Abhängigkeit von der Wellenlänge/Frequenz vorhanden ist. Bei Licht herrscht Dispersion bei der Brechung.
  \subsection{Fehlerrechnung}
  \begin{align}
    x=f(x_1,x_2,..,x_n)\\
    \Delta x = \sum_{i=1}^n{\left |\frac{\partial  x}{\partial x_i}\Delta x_i\right |}
    \label{error}
  \end{align}

\section{Versuchsaufbau}
\section{Geräteliste}
\begin{itemize}
  \item Refraktometer nach Abbe 
  \item Refraktometer nach Zeiss
  \item Flüssigkeit
  \item Natriumdampflampe
\end{itemize}

\section{Messung}
\subsection{Messen der Lotrichtungen vom Messprisma}
\begin{align}
  r_1 = \\
  r_2 = 
\end{align}

\subsection{Messen des Differenzwinkel zum Lot des Helligkeitsübergangs}
\begin{align}
  \gamma=
\end{align}

\subsection{Messen des Differenzwinkel zum Lot des Helligkeitsübergangs mit Flüssigkeit}
\begin{align}
  \gamma_f=
\end{align}

\subsection{Messung der Brechzahl mit dem Zeiss-Refraktometer}
\begin{align}
  n_{fz}=  
\end{align}

\section{Rechnung/Auswertung}
Fehlerrechnung nach \ref{error}
\subsection{Kantenwinkel des Messprismas}
\begin{align}
 \phi = 180-\left| r_1-r_2 \right|=\\
\end{align}

\subsection{Berechnung der Brechzahl des Messprismas}
\begin{align}
  n_g=\sqrt{\left(\frac{sin \gamma + cos \phi}{sin \phi}\right)^2+1}=
  \label{}
\end{align}

\subsection{Berechnung der Brechzahl der Flüssigkeit}
\begin{align}
  \alpha_f &= \phi - arcsin\left(\frac{sin(\gamma_f)}{n_g}\right)=\\
  n_f&=n_g sin(\alpha_f)= \\
\end{align}

\section{Zusammenfassung}
\end{document}




