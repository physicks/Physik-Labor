%     Created: Don Nov 25 12:00  2010 C
% Last Change: Don Nov 25 12:00  2010 C
%
\documentclass[a4paper]{article}
\usepackage[]{amsmath}
\usepackage{ngerman}
\usepackage[utf8]{inputenc}
\usepackage[]{graphicx}
\usepackage{fancyhdr}
\usepackage{lastpage}

\pagestyle{fancy}
\fancyhead{}
\fancyfoot{}
\fancyhead[L]{Armin Fürst}
\fancyhead[C]{Refraktometer}
\fancyhead[R]{G27}
\fancyfoot[C]{\thepage/\pageref{LastPage} }

\begin{document}
\section{Aufgabenstellung}
\begin{itemize}
  \item Bestimmen der Brechzahl einer Flüssigkeit mit dem Abbe-Refraktometer bei der Wellenlänge einer Natrium D-Linie.
  \item Überprüfen des Ergebnisses mit einem Zeiss Refraktometer.
\end{itemize}

\section{Voraussetzungen}
\subsection{Brechungsgesetz}
\begin{align}
  \frac{n_1}{n_2}=\frac{sin(\gamma_2)}{sin(\gamma_1)}
  \label{bechungsgesetz}
\end{align}
\subsection{Definition der Brechzahl}
\begin{align}
  n_m = \frac{c_0}{c_m}=\frac{sin(\gamma_0)}{sin(\gamma_m)}
  \label{brechzahl}
\end{align}
\subsection{Dispersion}
Dispersion tritt auf wenn eine Abhängigkeit von der Wellenlänge/Frequenz vorhanden ist. Bei Licht herrscht Dispersion bei der Brechung.
\subsection{Fehlerrechnung}
\subsubsection{Größtfehler} 
\begin{align}
  x=f(x_1,x_2,..,x_n) \quad
  \Delta x = \sum_{i=1}^n{\left |\frac{\partial  x}{\partial x_i}\Delta x_i\right |}
  \label{error}
\end{align}
\subsubsection{Gauss}
\begin{align}
  x = \sum_{i=1}^n{\frac{x_i}{n}} \quad
  \Delta x = \frac{1}{\sqrt{n}}\sqrt{\sum_{i=1}^n{\frac{(x_i-x)^2}{n-1}}}
\end{align}
\newpage
\section{Versuchsaufbau}
Verwendet wurde ein Zeiss- bzw ein Abbe-Refraktometer.
Beim Zeiss-Refraktometer wurde weißes Licht verwendet.
Die Natriumdampflampe wurde für das Abbe-Refraktometer verwendet.

\begin{figure}[ht]
  \begin{center}
    \includegraphics[width=0.5\textwidth]{abbe.png}
  \end{center}
  \caption{Abbe-Refraktometer. (1) Beleuchtungsprisma, (2) Messprisma, (3) Winkelskala
(Winkeleinteilung mit Nonius-Skala für Winkelminuten auf der Rückseite), (4) Fernrohr.}
  \label{fig:abbe}
\end{figure}

\begin{figure}[ht]
  \begin{center}
    \includegraphics[width=0.5\textwidth]{zeiss.png}
  \end{center}
  \caption{Zeiss-Refraktometer. (1) Beleuchtungsprisma (aufgeklappt), (2) Messprisma, (3)
Winkelverstellung, (4) Kompensator, (5) Okular.}
  \label{fig:zeiss}
\end{figure}

\section{Geräteliste}
\begin{itemize}
  \item Refraktometer nach Abbe 
  \item Refraktometer nach Zeiss
  \item Natriumdampflampe
  \item Glycerin
  \item Destilliertes Wasser
\end{itemize}
\newpage
\section{Messung}

\subsection{Messen der Lotrichtungen auf die Prismaflächen(Messprismas)}
Um den Brechenden Winkel des Messprismas zu bestimmen, werden die 2 Lotrichungen auf die Prismenflächen bestimmt. Dies geschieht durch das Suchen des reflektierten Lichtstrahls aus dem Umlenkprisma.
\begin{table}[ht]
  \caption{Messung der Lotrichtung}
  \centering
  \begin{itemize}
    \item r1 \dots Lotrichtung 1, U=$\pm 2'$
    \item r2 \dots Lotrichtung 2, U=$\pm 2'$
  \end{itemize}
  \begin{tabular}{|c|c|}
    \hline
    r1&r2\\
    \hline
    $-2.5^\circ 0'$	& $115.5^\circ 20'$ \\
    \hline
    $-2.5^\circ 28'$	& $114.5^\circ 22'$ \\
    \hline
    $-2.5^\circ 28'$	& $114.5^\circ 24'$ \\
    \hline
  \end{tabular}
  \label{tab:lot}
\end{table}

\subsection{Messen des Differenzwinkel zum Lot des Helligkeitsübergangs}
Um nachher die Brechzahl des Messprismas bestimmen zu können wird die Hell-Dunkel Grenze des Messprismas gemessen. Dazu wird das Beleuchtungsprima aufgesetzt aber keine Flüssigkeit aufgetragen.
\begin{table}[ht]
  \centering
  \caption{Maximaler Brechungswinkel bei Luft}
  \begin{itemize}
    \item $\gamma$ \dots Differenzwinkel, U = $\pm 2'$
  \end{itemize}
  \begin{tabular}{|c|}\hline
    $\gamma$\\ \hline
    $170^\circ 0'$\\\hline
    $170^\circ 0'$\\\hline
    $169.5^\circ 22'$\\\hline
    $169.5^\circ 26'$\\\hline
    $169.5^\circ 26'$\\\hline
    $169.5^\circ 26'$\\\hline
  \end{tabular}
  \label{tab:luft}
\end{table}
\newpage
\subsection{Messen des Differenzwinkel zum Lot des Helligkeitsübergangs mit Flüssigkeit}
Nun wird um später die Brechzahl der Flüssigkeiten zu bestimmen der Maximale Brechungswinkel gesucht. Dazu wird ein Flüssigkeitsfilm zwischen Messprisma und Beleuchtungsprisma erzeugt bevor die Grenze gemessen wird.
\subsubsection{Glycerin}
\begin{table}[ht]
  \centering
  \caption{Maximaler Brechungswinkel bei Glycerin}
  \begin{itemize}
    \item $\gamma_{f1}$\dots Maximaler Brechungswinkel bei Glycerin,U=$\pm2'$
  \end{itemize}
  \begin{tabular}{|c|}\hline
    $\gamma_{f1}$\\\hline
    $126.5^\circ 0'$\\	\hline
    $126.0^\circ 26'$\\	\hline
    $126.0^\circ 28'$\\	\hline
  \end{tabular}
  \label{tab:glycerin}
\end{table}
\subsubsection{Wasser}
\begin{table}[ht]
  \centering
  \caption{Maximaler Brechungswinkel bei Wasser}
  \begin{itemize}
    \item $\gamma_{f2}$\dots Maximaler Brechungswinkel bei Wasser,U=$\pm2'$
  \end{itemize}
  \begin{tabular}{|c|}\hline
    $\gamma_{f2}$\\\hline
    $138.0^\circ 12'$\\	\hline
    $138.0^\circ 14'$\\	\hline
    $138.0^\circ 10'$\\	\hline
  \end{tabular}
  \label{tab:wasser}
\end{table}

\subsection{Messung der Brechzahl mit dem Zeiss-Refraktometer}
Die Brechzahl kann bei diesem Refraktometer direkt abgelesen werden.
\subsubsection{Glycerin}
\begin{align}
  n_f=1.456\pm 0.001				
\end{align}
\subsubsection{Wasser}
\begin{align}
  n_f2=1.336\pm 0.001				
\end{align}
\newpage
\section{Rechnung/Auswertung}
\subsection{Kantenwinkel des Messprismas}
\begin{align}
  \phi = 180-\left| r_1-r_2 \right|
\end{align}
\begin{table}[ht]
  \centering
  \caption{Brechender Winkel des Messprismas}
  \begin{itemize}
    \item $\phi$ \dots Brechender Winkel
    \item $\Delta \phi$ \dots Größtfehler Brechender Winkel
  \end{itemize}
  \begin{tabular}{|c|c|}\hline
    $\phi/Grad$&$\Delta \phi/Grad$\\\hline
    61.6667 & 0.0666667 \\\hline
    63.1 & 0.0666667 \\\hline
    63.0667 & 0.0666667\\\hline
  \end{tabular}
  \label{tab:brech}
\end{table}
\begin{align}
  \text{Mittelwert nach Gauss:}\quad
  \phi = (62.6 \pm 0.5)^\circ
\end{align}


\subsection{Berechnung der Brechzahl des Messprismas}
\begin{align}
  n_g=\sqrt{\left(\frac{sin \gamma + cos \phi}{sin \phi}\right)^2+1}
\end{align}
\begin{table}[ht]
  \centering
  \caption{Brechzahl des Messprismas}
  \begin{itemize}
    \item $n_g$ \dots Brechzahl
    \item $\Delta n_g$ \dots Größtfehler der Brechzahl
  \end{itemize}
  \begin{tabular}{|c|c|}\hline
    $n_g$&$\Delta n_g$\\\hline
    1.75186 & 0.0124331 \\\hline
    1.75186 & 0.0124331 \\\hline
    1.75062 & 0.0124255 \\\hline
    1.75124 & 0.0124293 \\\hline
    1.75124 & 0.0124293 \\\hline
    1.75124 & 0.0124293 \\\hline
  \end{tabular}
  \label{tab:brechzahlglas}
\end{table}
\begin{align}
  \text{Mittelwert nach Gauss:}\quad
  n_g = 1.7513\pm0.0002 
\end{align}

\subsection{Berechnung der Brechzahl der Flüssigkeiten}
\begin{align}
  n &=n_g sin\left( \phi - arcsin\left(\frac{sin(\gamma_f)}{n_g}\right)\right)\\
\end{align}
$\gamma_f$ entspricht bei Wasser $\gamma_{f2}$ und bei Glycerin $\gamma_{f1}$
\begin{table}[ht]
  \centering
  \caption{Brechungsindex der Flüssigkeiten}
  \begin{itemize}
    \item $n_{f1}$ \dots Brechungsindex von Glycerin
    \item $\Delta n_{f1}$ \dots Größtfehler Brechungsindex von Glycerin
    \item $n_{f2}$ \dots  Brechungsindex von Wasser
    \item $\Delta n_{f2}$ \dots Größtfehler Brechungsindex von Wasser
  \end{itemize}
  \begin{tabular}{|c|c|c|c|}\hline
    $n_{f1}$&$\Delta n_{f1}$&$n_{f2}$&$\Delta n_{f2}$\\\hline
    1.45512 & 0.0198069 & 1.33609 & 0.0213668 \\\hline
    1.4516 & 0.0198328 & 1.33644 & 0.0213649 \\\hline
    1.45544 & 0.0198045 & 1.33573 & 0.0213686 \\\hline
  \end{tabular}
  \label{tab:flue}
\end{table}
\begin{align}
  \text{Brechzahl von Glycerin nach Gauss:}\quad n_f&=1.454\pm0.001\\
  \text{Brechzahl von Wasser nach Gauss:}\quad n_f2&=1.3360 \pm 0.0002	
\end{align}

\section{Zusammenfassung}
Durch die Messung haben sich folgende Werte für die Brechzahlen ergeben:
\begin{align}
  \text{Brechzahl von Glycerin Abbe:}\quad n_f&=1.45\pm0.02\\
  \text{Brechzahl von Glycerin Zeiss:}\quad n_f&=1.456\pm0.001\\
  \text{Brechzahl von Wasser Abbe:}\quad n_f2&=1.34 \pm 0.02	\\
  \text{Brechzahl von Wasser Zeiss:}\quad n_f2&=1.336 \pm 0.001	
\end{align}

Die mit dem Abbe-Refraktometer gemessenen Werte stimmen gut mit denen des Zeiss-Refraktometers überein. Die Messabweichungen des Brechenden Winkels kam daher , dass die Apperatur bei dieser Messung sehr empfindlich reagierte.

\end{document}




