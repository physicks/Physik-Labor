%        File: template.tex
%     Created: Don Nov 25 12:00  2010 C
% Last Change: Don Nov 25 12:00  2010 C
%
\documentclass[a4paper]{article}
\usepackage[]{amsmath}
\usepackage{ngerman}
\usepackage[latin1]{inputenc}
\usepackage[]{graphicx}
\usepackage{fancyhdr}
\usepackage{epstopdf}
\usepackage{lastpage}

\pagestyle{fancy}
\fancyhead{}
\fancyfoot{}
\fancyhead[L]{Armin F�rst \linebreak Michael Pusterhofer}
\fancyhead[C]{Interferometer}
\fancyhead[R]{G27}
\fancyfoot[C]{\thepage/\pageref{LastPage} }

\begin{document}
\section{Aufgabenstellung}
\begin{itemize}
  \item Kalibrierung des Kompensators mit monochromatischen Licht der Natriumdampflampe mit einer Wellenl�nge von 589nm.
  \item	Bei Wei�licht die Interferenzmuster zur Deckung bringen und Kompensatorwert ablesen. (Nullpunkteinstellung)
  \item	Einf�llen des Probegases (Stickstoff), Interferenzmuster zur Deckung bringen und Kompensatorwert ablesen.
  \item Brechzahlberechnung von Stickstoff mit dem gemessenen Wegl�ngenunterschied.
\end{itemize}

\section{Voraussetzungen}
L ist die (mechanische) L�nge der gleichartig gefertigten Mess- und Referenzkammer.
Sind sie mit (unterschiedlichen) Gasen der Brechzahlen $n_m$ , $n_r$ gef�llt und die
Lichtwellen durchlaufen in ihnen die optischen Wegl�ngen:
\begin{eqnarray}
l_m=n_m L \\
l_r=n_r L
\end{eqnarray}

Zwischen den Wellen entsteht somit der Gangunterschied l:
\begin{align}
l=|l_m-l_r|=L|n_m-n_r|
\end{align}

Wird monochromatisches Licht verwendet und der Gangunterschied durch Drehen des
Kompensators vom Skalenwert $X_1$ nach $X_2$ um $l=N*\lambda_0$ ge�ndert ($\lambda_0$
 ist die Vakuumwellenl�nge des verwendeten Lichtes), verschiebt sich das Interferenzmuster
um N Interferenzstreifen. Daraus l�sst sich durch Messung der Kompensatorstellung
bei unterschiedlichen N der Zusammenhang zischen Kompensatorstellung und
Gangunterschied bestimmen (Ausgleichsgerade mit Steigung $\beta$):
\begin{align}
l=N\lambda_0=\lambda_0\beta(X_2-X_1)
\end{align}

Unter Verwendung von Wei�licht bestimmt man nun die zum Ausgleich des durch
die F�llung der Messkammer mit Messgas erzeugten Gangunterschiedes notwendige
�nderung der Kompensatorstellung X. Daraus l�sst sich die Brechzahldifferenz
berechnen:
\begin{eqnarray}
|n_m-n_r|=\frac{\lambda_0\beta X}{L} \\
n_m=n_r+\frac{\lambda_0\beta X}{L}
\end{eqnarray}
Gr��tfehler f�r $n_m$ mit Fehler f�r X und $\beta$:
\begin{align}
\Delta n_m=|\frac{\lambda_0\beta}{L}|*\Delta X + |\frac{\lambda_0 X}{L}|*\Delta\beta
\end{align}

\clearpage
\section{Versuchsaufbau}
	\begin{figure}[h]
			\centering
			\includegraphics{interferometer.jpg}
			\caption{Interferometer nach Arago L�we, Sp=Spalt, Kl=Kolimator, Bl=Doppelblende, L=Luftkammer, G=Kammer f�r Messgas, Pl=drehbare Kompensatorplatte, Pg=feste Kompensatorplatte,
			H=Hilfsplatte, F=Fernrohr, Ok=Okular, Tr=Messtrommel, A und B = Gaszu- und abf�hrungsstutzen.}
			\label{fig:aufbau_interferometer}
	\end{figure}
	
	
\section{Ger�teliste}
\begin{table}[ht]
		\caption{Ger�teliste}
		\centering
		\begin{tabular}{c|c}
			Ger�t & Inventarnummer \\\hline \hline
			Interferometer nach Arago-L�we & - \\
			Monochromatische Lichtquelle (Na-Spektrallampe) & - \\
			Wei�lichtquelle (Halogenlampe) & - \\
			Ballon mit Stickstoff & - \\
			\end{tabular}
	\label{tab:list_tools}
\end{table}

\clearpage

\section{Messung}
	\subsection{Kalibrierung des Kompensators mit 589nm (Na-Spektrallampe)}
		\begin{table}[ht]
				\caption{Messen der Kalibrierungskurve des Kompensators}
				\begin{itemize}
					\item M \dots  Messnummer
					\item X \dots  Gangunterschied, $\Delta$X = $\pm$0.01
					\item $N_I$ \dots  Anzahl der Inteferenzstreifen um die verschoben wurde, $\Delta N_I$ = $\pm$1
				\end{itemize}
				\centering
				\begin{tabular}{c|c|c}
				 M & X & $N_I$ \\\hline \hline
				 1 & -0.16 & 0 \\\hline
				 2 & 0.15 & 1 \\\hline
				 3 & 0.43 & 2 \\\hline
				 4 & 0.74 & 3 \\\hline
				 5 & 1.04 & 4 \\\hline
				 6 & 1.34 & 5 \\\hline
				 7 & 1.64 & 6 \\\hline
				 8 & 1.93 & 7 \\\hline
				 9 & 2.23 & 8 \\\hline
				 10 & 2.52 & 9 \\\hline
				 11 & 2.81 & 10 \\\hline
				 12 & 3.10 & 11
				\end{tabular}
			\label{tab:list_diode}
		\end{table}

	\subsection{Nullpunkt bei Wei�licht (Kompensatorwert mit Luft in beiden Kammern und deckungsgleichen Interferenzmustern)}
	\begin{center}
	$W_0 = -0.72$ \\
	$\Delta W_0= \pm0.01$ \\	
	\end{center}
	
	\subsection{Kompensatorwert mit Stickstoff (mit Stickstoff in einer Kammer und deckungsgleichen Interferenzmustern}
	\begin{center}
	$N_K = 2.06$ \\
	$\Delta N_K= \pm0.01$ \\	
	\end{center}
\clearpage

\section{Rechnung/Auswertung}
	\subsection{Kalibrierungskurve}
		\begin{figure}[ht]
			\centering
			\includegraphics{if_kalibrierungskurve.eps}
			\caption{Kalibrierungskurve des Kompensators. Interferenzstreifen $N_I$ �ber den Gangunterschied X eingestellt am Kompensator. 
			Die gro�en Fehlerbalken in Y-Richtung kommen daher, dass ein Fehler von 1 bei den Interferenzstreifen angenommen wurde. Da die Fehler in X-Richtung nur 0.01 betragen sieht man die Fehlerbalken kaum.}
			\label{fig:kalibrierungskurve}
		\end{figure}
		
		\subsubsection{Mathematica Fit (LinearModelFit)}
			\begin{center}
			$N_I=\beta X+d$ \\
			\end{center}
			\begin{center}
			$d=(0.51 \pm0.02)$ \\
			$\beta=(3.370 \pm0.008)$ \\
			\end{center}
			
	\subsection{Brechungsindex von Stickstoff nach Gl.6 und Gl.7}
		\begin{flushleft}
			$n_L=1.0002926$ \dots Literaturwert\\
			$\lambda_0=589*10^{-9}$ \dots Na-Spektrallinie\\
			$L=1m$ \dots Mechanische L�nge der Mess und Referenzkammer \\
			$X=N_K-W_0=(2.78 \pm 0.02)$\\
		\end{flushleft}
		\begin{center}
			$n_N=1.000298119$ \\
			$\Delta n_N=53*10^{-9}$
		\end{center}
	
\clearpage
\section{Zusammenfassung}
	\subsection{Kalibrierungskurve}
	Die aufgenommene Kalibrierungskurve des Kompensators kommt einer Gerade sehr nahe. 
	\subsection{Brechnungindex von Stickstoff}
		\begin{center}
			$n_N=(1.00029812 \pm6*10^{-8})$
		\end{center}
		Der gemessene Wert f�r Stickstoff stimmt sehr gut mit dem Literaturwert �berein.
\end{document}