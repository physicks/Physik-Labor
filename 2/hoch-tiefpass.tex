%        File: hoch-tiefpass.tex
%     Created: Mon Nov 22 06:00  2010 C
% Last Change: Mon Nov 22 06:00  2010 C
%
\documentclass[a4paper]{article}
\usepackage[]{amsmath}
\usepackage[utf8]{inputenc}
\usepackage[]{graphicx}
\usepackage{ngerman}
\begin{document}

\section{Aufgabenstellung}
\begin{itemize}
  \item Messen des Frequenz-/Phasengangs eines RC Hochpasses
  \item Messen des Frequenz-/Phasengangs eines RC Tiefpasses
  \item Berechnen der Grenzfrequenz und der Kapazität
  \item Errechnen der Kapazität durch Messen des Spannungsverlaufs eines RC glieds als Integrator
  \item Errechnen der Kapazität durch Messen des Spannungsverlaufs eines RC glieds als Differentiator
\end{itemize}

\section{Voraussetzungen}
  Der Widerstand des RC Glieds kann komplex als die Summe der Impedanzen dargestellt werden. Da die Impedanz des Kondensators frequenzabhängig ist hat man je nach Frequenz unterschiedliches Verhalten der Schaltung(frequenzabhängiger Spannungsteiler). 

  Aus der komplexen Rechnung folgen nun Folgende Formeln

  Grenzfrequenz eines RC Glieds

  \begin{align}
    \nu_g = \frac{1}{2 \pi R C} \quad
    \Delta \nu_g = \left | \frac{-1}{2 \pi R^2 C} \right | \Delta R + \left | \frac{-1}{2 \pi R C^2}\right | \Delta C
  \end{align}

Beim Tiefpass lässt sich bei großen Frequenzen folgende Nähnerung für die Kapazität angeben
\begin{align}
  C = \frac{U_E}{R}\frac{dt}{d U_A}  
\end{align}

Für den Hochpass lässt sich bei kleinen Frequenzen folgende Näherung für die Kapazität machen.
\begin{align}
  C=\frac{U_A}{R}\frac{dt}{d U_E}  
\end{align}

\section{Versuchsaufbau}
Bei allen Versuchen verwenden wir ein RC Serienglied.
CH1 des Oszilloskops wird jeweils über das ganze RC Glied gemessen.
CH2 wird an jenem Element gemessen am den die Ausgangsspannung anliegt.
\subsection{Hochpass / Differentiator}
  Die Ausgangsspannung wird am Widerstand abgenommen. Die Masse des Oszilloskops befindet sich am Widerstand auf der Seite der Versorgung.
\subsection{Tiefpass/Integrator}
  Die Ausgangsspannung wird am Kondensator abgenommen. Die Masse des Oszilloskops befindet sich am Kondensator auf der Seite der Versorgung.

\section{Geräteliste}
\begin{itemize}
  \item Oszilloskop Hameg
  \item USB Oszilloskop DSO220
  \item Funktionsgenerator Hameg
  \item Widerstand $(1800\pm2\text{\%})\Omega$
  \item Kapazität $(0.1\pm10\text{\%})\mu$F
\end{itemize}

\section{Messungen}
Die Eingangsamplitude beträgt jeweils 4V.
\subsection{Hochpass}
Es werden bei unterschiedlichen Frequenzen der Phasenunterschied und die Amplitude der Ausgangsspannung gemessen.

\begin{table}[ht]
  \centering
  \caption{Ausgangsamplutide und Phasenunterschied abhängig von der Frequenz}
  \begin{itemize}
    \item f \dots Frequenz
    \item $\phi$ \dots Phasenunterschied
    \item $U_A$ \dots Ausgangsamplitude
  \end{itemize}
  \begin{tabular}{|c|c|c|}
    \hline
    f/Hz&$\phi$/ms&$U_A$/V\\
    \hline
    5.867&40&0.280\\
    \hline
    10.21&24&0.480\\
    \hline
    66.5&3.6&0.3\\
    \hline
    140.0&1.6&0.6\\
    \hline
    209.1&1&0.9\\
    \hline
    326.0&0.6&1.4\\
    \hline
    487.8&0.35&2\\
    \hline
    565.2&0.28&2.2\\
    \hline
    1107&0.1&3\\
    \hline
    1878&0.04&3.6\\
    \hline
    2751&0.02&3.7\\
    \hline
    3886&0.004&3.8\\
    \hline
    5030&0.06&4\\
    \hline
    11790&0.001&4\\
    \hline
    20160&0.0004&4\\
    \hline
  \end{tabular}
  \label{tab:1}
\end{table}
\newpage
\subsection{Tiefpass}
\begin{table}[ht]
  \centering
  \caption{Ausgangsamplitude und Phasenunterschied abhängig von der Frequenz}
  \begin{itemize}
    \item f \dots Frequenz
    \item $\phi$ \dots Phasenunterschied
    \item $U_A$ \dots Ausgangsamplitude
  \end{itemize}
  \begin{tabular}{|c|c|c|}
    \hline
    f/Hz&$\phi$/ms&$U_A$/V\\
    \hline
    15,00&0&4\\
    \hline
    57.27&0.2&4\\
    \hline
    105.9&0.2&4\\
    \hline
    266.8&0.2&3.8\\
    \hline
    402.5&0.2&3.6\\
    \hline
    565.2&0.16&3.3\\
    \hline
    1052&0.130&2.5\\
    \hline
    2144&0.090&2\\
    \hline
    3543&0.060&0.9\\
    \hline
    4760&0.048&0.7\\
    \hline
    5311&0.044&0.6\\
    \hline
    5790&0.040&0.56\\
    \hline
    10460&0.022&0.32\\
    \hline
    19610&0.012&0.16\\
    \hline
  \end{tabular}
  \label{tab:2}
\end{table}



\newpage
\subsection{Integrator}

Aufnahme des Oszilloskopbildes mit dem USB Oszilloskops
\begin{figure}[ht]
\begin{center}
  \includegraphics[width=0.8 \textwidth]{int.eps}
\end{center}
\caption{RC-Glied im Integratorbetrieb}
\label{fig:5}
\end{figure}

\subsection{Differentiator}

Aufnahme des Oszilloskopbildes mit dem USB Oszilloskops
\begin{figure}[ht]
  \begin{center}
    \includegraphics[width=0.8 \textwidth]{dif.eps}
  \end{center}
  \caption{RC Glied im Differentiatorbetrieb}
  \label{fig:6}
\end{figure}


\newpage
\section{Rechung/Auswertung}
\subsection{Hochpass}

\begin{figure}[ht]
  \begin{center}
    \includegraphics[width=0.8 \textwidth]{uehoch.eps}
  \end{center}
  \caption{Übertragungsfunktion des Hochpasses}
  \label{fig:1}
\end{figure}

\begin{figure}[ht]
  \begin{center}
    \includegraphics[width=0.8 \textwidth]{phhoch.eps}
  \end{center}
  \caption{Phasendiagramm des Hochpasses}
  \label{fig:2}
\end{figure}
\newpage
\subsection{Tiefpass}
\begin{figure}[ht]
  \begin{center}
    \includegraphics[width=0.8 \textwidth]{uetief.eps}
  \end{center}
  \caption{Übertragungsfunktion des Tiefpasses}
  \label{fig:3}
\end{figure}

\begin{figure}[ht]
  \begin{center}
    \includegraphics[width=0.8 \textwidth]{phtief.eps}
  \end{center}
  \caption{Phasendiagramm des Tiefpasses}
  \label{fig:4}
\end{figure}

\subsection{Berechnung von Grenzfrequenz und Kapazität aus den Bodediagrammen}

Aus dem Frequenzgang kann man jeweils bei einer Dämpfung von ca 0.7 die Grenzfrequenz ablesen.
Bei dem Phasendiagramm geht dies bei 45 Grad.
\newpage
Daraus Resultieren Folgende Werte
\begin{table}[ht]
  \centering
  \caption{Grenzfrequenz aus den Bodediagrammen}
  \begin{tabular}{|r|c|c|}
    \hline
    &Frequenzgang&Phasendiagramm \\
    \hline
    Hochpass&825$\pm$5&855$\pm$5\\
    \hline
    Tiefpass&895$\pm$5&865$\pm$5\\
    \hline
  \end{tabular}
  \label{tab:3}
\end{table}

Mittelt man diese Werte erhält man 
\begin{align}
  \nu_g = (860 \pm 30)Hz
\end{align}

Aus den Bauelementen kann man über Gleichung 1 ebenfalls die Grenzfrequenz errechnen.
\begin{align}
  \Delta R = 36 \Omega \quad \Delta C= 0.01 \mu F\\
  \nu_g = (800\pm100)Hz  
\end{align}

Mit der Grenzfrequenz aus den Bodediagrammen errechnen wir nun die Kapazität mit Hilfe der Gleichung 1(Der Fehler wird mit der Größtfehlermethode gerechnet).
\begin{align}
  C =\frac{1}{2 \nu \pi R}=(0.103 \pm 0.004)\mu F  
\end{align}

\subsection{Integrator}


Über die Gleichung 2 lässt sich die Kapazität berechnen.
\begin{align}
  \frac{dU_A}{dt} = 2 \frac{V}{ms} \quad U_E = 2 V \quad R=1800 \Omega\\ 
  \Rightarrow C = 0.5 \mu F
\end{align}

\subsection{Differentiator}
Über die Gleichung 3 lässt sich die Kapazität berechnen.
\begin{align}
  \frac{dU_E}{dt} = 380 \frac{V}{s} \quad U_A = 0.13 mV \quad R=1800 \Omega\\ 
  \Rightarrow C = 0.2 \mu F
\end{align}


\section{Zusammenfassung}
Über Frequenzgang und Phasengang lässt sich die Kapazität/Grenzfrequenz relativ schön errechnen.

\begin{align}
  C =(0.103 \pm 0.004)\mu F\\
  \nu_g = (855\pm20)Hz  
\end{align}
Beim Differentiator/Integratorbetrieb stimmt zwar die Größenordnung aber der Wert liegt daneben.

\begin{align}
  C_{INT} = 0.5 \mu F\\
  C_{DIFF} = 0.2 \mu F
\end{align}
Dies ist möglicherweise darauf zurückzuführen, dass die Oszidaten digital aufgenommen und diese Dateien in einem mir unbekannten Format(dso) gespeichert wurden. Ich habe versucht die Daten möglichst richtig zu Plotten, diese haben aber dadurch womöglich einen systematischen Fehler erhalten.

\end{document}



