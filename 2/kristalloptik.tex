%        File: kristalloptik.tex
%     Created: Mon Nov 22 03:00  2010 C
% Last Change: Mon Nov 22 03:00  2010 C
%
\documentclass[a4paper]{article}
\usepackage[]{amsmath}
\usepackage[utf8]{inputenc}
\usepackage{graphicx}
\usepackage{ngerman}
\begin{document}

\section{Aufgabensetllung}
\begin{itemize}
\item
  Es ist experimentell das Gesetz von Malus zu beweisen.
\item
  Es ist das Verhalten von $\frac{\lambda}{2}$-/ $\frac{\lambda}{4}$- Plättchen zu messen.
\end{itemize}

\section{Voraussetzungen}
\subsection{Gesetz von Malus}
Allgemein.
\begin{align}
  I = c * E^2 
\end{align}
Bei linearer Polarisierung mit dem Winkel $\theta$ gilt
\begin{align}
  I = c * E^2 * cos^2(\theta) 
\end{align}
Aus dem Verhältnis der maximalen Intensität und der Gemessenen bekommt man das Gesetz von Malus
\begin{align}
    \frac{I}{I_{max}}= \cos^2{\theta}
\end{align}
\subsection{$\frac{\lambda}{x}$-Plättchen}
Aufgeteilt in x und y Richtung kann man E wie folgt schreiben
\begin{align}
  E_x &= E cos(\psi) cos(\omega t - \phi)\\
  E_y &= E sin(\psi) sin(\omega t - \phi - \frac{\lambda}{x})
\end{align}
wobei $\phi$ die Drehung beim durchlaufen des Plättchens beschreibt und $\frac{\lambda}{x}$ die relative Verschiebung zur 2. Richtungen ist. $\psi$ ist der Winkel der Polarisierungsrichtung zur optischen Hauptachse.
Bei x = 2 wird das Licht um 2 $\psi$ gedreht und bei x = 4 entsteht zirkular polarisiertes Licht.

\section{Versuchsaufbau}
Nach der Reihe wurden folgende Objekte auf einer Schiene montiert.
\begin{itemize}
\item Laser
\item Polarisator
\item $\frac{\lambda}{2}$ oder $\frac{\lambda}{4}$ Plättchen
\item Analysator
\item Photodetektor mit Laser-Linenfilter
\end{itemize}
Je nach Versuch wurde das $\frac{\lambda}{2}$ , $\frac{\lambda}{4}$ Plättchen sowie der Analysator entfernt.
Der Photodetektor wurde an ein Voltmeter angeschlossen um die Intensität zu messen.

\section{Geräteliste}
\begin{itemize}
  \item 670nm Laser
  \item Polariator in Drehhalterung, U=$\pm 2^\circ$
  \item $\frac{\lambda}{2}$ und $\frac{\lambda}{4}$ Plättchen in Drehhalterung, U=$\pm 2^\circ$
  \item Analysator in Drehhalterung, U=$\pm 2^\circ$
  \item Photodetektor mit Laser-Linenfilter
  \item Voltcraft Voltmeter VC220, U=$\pm(0.5\% + 2 dgts)$
\end{itemize}

\section{Messung}
\subsection{Kalibierung}
  \subsubsection{Messung der Spannung am Photodetektor mit ausgeschltetem Laser}
  \begin{align}
    U_D=(-6.2 \pm 0.1)mV
  \end{align}
  
  \subsubsection{Messung der Spannung am Photodetektor bei eingeschaltetem Laser und Polarisator auf $0^\circ$(nur mit Polaristor zwischen Laser und Detektor)}
  \begin{align}
    U_D=(7.94 \pm 0.05)V
  \end{align}
\newpage
\subsection{Messung der Spannung am Photodetektor bei Veränderung des Analysatorwinkels(Polarisator auf $0^\circ$ und ohne $\lambda$-Plättchen)}

\begin{table}[ht]
  \centering
  \caption{Gesetz von Malus}
  \begin{itemize}
    \item N \dots Messnummer
    \item U \dots Spannung am Photodetektor
    \item $\Delta U$ \dots Fehler der Spannung am Photodetektor
    \item $\phi_A$ \dots Anlaysatorwinkel, $U=\pm2^\circ$
  \end{itemize}
  \begin{tabular}{|c|c|c|c|}
    \hline
    N&U/V&$\Delta U$/V&$\phi_A/Grad$\\
    \hline
    1&5.56&0.04&0\\
    \hline
    2&5.55&0.04&5\\
    \hline
    3&5.33&0.04&10\\
    \hline
    4&5.09&0.04&15\\
    \hline
    5&4.72&0.03&20\\
    \hline
    6&4.33&0.03&25\\
    \hline
    7&3.86&0.03&30\\
    \hline
    8&3.41&0.03&35\\
    \hline
    9&2.91&0.03&40\\
    \hline
    10&2.42&0.02&45\\
    \hline
    11&1.910&0.02&50\\
    \hline
    12&1.480&0.008&55\\
    \hline
    13&1.047&0.06&60\\
    \hline
    14&0.701&0.005&65\\
    \hline
    15&0.417&0.003&70\\
    \hline
    16&0.1983&0.001&75\\
    \hline
    17&0.0540&0.0004&80\\
    \hline
    18&0.0000&0.0001&85\\
    \hline
    19&0.0257&0.0002&90\\
    \hline
    20&0.1312&0.0008&95\\
    \hline
    21&5.66&0.04&-5\\
    \hline
  \end{tabular}
  \label{tab:1}
\end{table}

\newpage
\subsection{$\frac{\lambda}{2}$-Plättchen}
\subsubsection{Messung des Intensitätsmaximums/minimums bei Analysator-/Polarisatorwinkel auf 0 Grad und der $\frac{\lambda}{2}$-Platte}
Minimum:
\begin{align}
\phi_{\frac{\lambda}{2}}&=(46 \pm 2)^\circ\\
U_D &= (12.3 \pm 0.2)mV
\end{align}
Maximum:
\begin{align}
\phi_{\frac{\lambda}{2}}&=(90\pm2)^\circ\\
U_D &= (5.45 \pm 0.04)V
\end{align}
\subsubsection{Messung des Intensitätsmaximums/minimums bei festen $\phi_{\frac{\lambda}{2}}$}
$\phi_{\frac{\lambda}{2}}= 30^\circ$
\begin{align}
\phi_{A_{max}}&=(60\pm 2)^\circ \quad U_{D_{max}}=(5.48\pm0.04)V  \\
\phi_{A_{min}}&=(328\pm 2)^\circ \quad U_{D_{min}}=(11.6\pm0.2)mV  
\end{align}
$\phi_{\frac{\lambda}{2}}= 15^\circ$
\begin{align}
\phi_{A_{max}}&=(28 \pm 2)^\circ \quad U_{D_{max}}=(5.57\pm0.04)V \\ 
\phi_{A_{min}}&=(299 \pm 2)^\circ \quad U_{D_{min}}=(2.6\pm 0.1)mV  
\end{align}
$\phi_{\frac{\lambda}{2}}= 0^\circ$
\begin{align}
\phi_{A_{max}}&=(0\pm2)^\circ \quad U_{D_{max}}=(5.55 \pm 0.04)V  \\
\phi_{A_{min}}&=(267\pm 2)^\circ \quad U_{D_{min}}=(0.05 \pm 0.01)mV  
\end{align}
\newpage

\subsection{Messung der Intensität beim einem $\frac{\lambda}{4}$-Plättchens bei unterschiedlichen Analysatorwinkeln}

\begin{table}[ht]
  \centering
  \caption{Intensität des $\frac{\lambda}{4}$-Plättchens}
  \begin{itemize}
    \item $\phi_A$ \dots Analysatorwinkel, U=$\pm 2^\circ$
    \item $U_{45}$ \dots Spannung am Detektor bei $45^\circ$ des $\frac{\lambda}{4}$-Plättchens, U=$\pm 0.02$V
    \item $U_{30}$ \dots Spannung am Detektor bei $30^\circ$ des $\frac{\lambda}{4}$-Plättchens
    \item $\Delta U_{30}$ \dots Fehler der Spannung am Detektor bei $30^\circ$ des $\frac{\lambda}{4}$-Plättchens
    \item $U_{15}$ \dots Spannung am Detektor bei $15^\circ$ des $\frac{\lambda}{4}$-Plättchens
    \item $\Delta U_{15}$ \dots Fehler der Spannung am Detektor bei $15^\circ$ des $\frac{\lambda}{4}$-Plättchens
    \item $U_{0}$ \dots Spannung am Detektor bei $0^\circ$ des $\frac{\lambda}{4}$-Plättchens
    \item $\Delta U_{0}$ \dots Fehler der Spannung am Detektor bei $0^\circ$ des $\frac{\lambda}{4}$-Plättchens
  \end{itemize}
  \begin{tabular}{|c|c|c|c|c|c|c|c|}
    \hline
    $\phi_A$/Grad&  $U_{45}$/V& $U_{30}$/V& $\Delta U_{30}$/V& $U_{15}$/V& $\Delta U_{15}$/V& $U_{0}$/V& $\Delta U_{0}$/V\\
    \hline
   0&2.48&2.85&0.02&4.21&0.03&5.22&0.04\\
    \hline
   10&2.33&3.13&0.03&4.53&0.03&5.11&0.04\\
    \hline
   20&2.21&3.33&0.03&4.60&0.03&4.71&0.03\\
    \hline
   30&2.13&3.44&0.03&4.43&0.03&4.04&0.03\\
    \hline
   40&2.11&3.45&0.03&4.04&0.03&3.32&0.03\\
    \hline
   50&2.14&3.36&0.03&3.51&0.03&2.29&0.02\\
    \hline
   60&2.24&3.18&0.03&2.86&0.02&1.45&0.02\\
    \hline
   70&2.40&2.95&0.02&2.18&0.02&0.74&0.01\\
    \hline
   80&2.55&2.67&0.02&1.56&0.02&0.26&0.01\\
    \hline
   90&2.71&2.38&0.02&1.07&0.02&0.05&0.01\\
    \hline
   100&2.87&2.12&0.02&0.76&0.01&0.15&0.01\\
    \hline
   110&2.99&1.90&0.02&0.67&0.01&0.54&0.01\\
    \hline
   120&3.06&1.80&0.02&0.82&0.01&1.17&0.02\\
    \hline
   130&3.10&1.78&0.02&1.18&0.02&2.00&0.02\\
    \hline
   140&3.07&1.85&0.02&1.71&0.02&2.89&0.02\\
    \hline
   150&2.97&2.03&0.02&2.35&0.02&3.76&0.03\\
    \hline
   160&2.82&2.27&0.02&3.02&0.03&4.47&0.03\\
    \hline
   170&2.65&2.63&0.02&3.64&0.03&4.96&0.03\\
    \hline
   180&2.47&2.81&0.02&4.12&0.03&5.14&0.04\\
    \hline
  \end{tabular}
  \label{tab:2}
\end{table}
\newpage
\section{Rechnung/Auswertung}
\subsection{Gesetz von Malus}
  Die Daten wurden in Mathematica geplottet und es ist zu erkennen dass man einen Offset des Analysatorwinkels von 4 Grad hat, da bei diesem Winkel die theoretischen mit den gemessenen Daten zusammenfallen

  \begin{figure}[ht]
    \begin{center}
      \includegraphics[width=0.8\textwidth]{malus.eps}
    \end{center}
    \caption{Gesetz von Malus}
    \label{fig:1}
  \end{figure}

  \subsection{$\frac{\lambda}{2}$-Plättchen}
  Wenn man jeweils das Minimum und das Maximum über den Winkel des Plättchens aufträgt erkennt man dass diese jeweils 90 Grad auseinanderliegen und um 2 mal den Winkel des Plättchens gedreht wurden.

  \begin{figure}[ht]
    \begin{center}
      \includegraphics[width=0.8\textwidth]{lambda2.eps}
    \end{center}
    \caption{$\frac{\lambda}{2}$-Plättchen}
    \label{fig:2}
  \end{figure}

  \subsection{$\frac{\lambda}{4}$-Plättchen}
  Da das Plättchen scheinbar nicht auf die Wellenlänge des Lasers angepasst ist erhalten wir kein zirkular polarisiertes Licht, sondern ein elliptisch polarisiertes Licht. Man kann die Spannung des Detektors über den Winkel des Analysators auftragen und erhält dann folgende Grafik. Die 4 Kurven unterscheiden sich im Winkel der Drehhalterung des Plättchens.

  \begin{figure}[ht]
    \begin{center}
      \includegraphics{lambda4.eps}
    \end{center}
    \caption{$\frac{\lambda}{4}$-Plättchen}
    \label{fig:3}
  \end{figure}

  \section{Zusammenfassung}

  Alle Versuche außer jener mit dem $\frac{\lambda}{4}$-Plättchens zeigten dass die theoretischen Werte gut mit der Realität zusammenpassen wie man in Abbildung 1 und Abbildung 2 gut sehen kann. Bei dem $\frac{\lambda}{4}$-Plättchen kann man durch die Theorie zeigen dass die gemessenen Figuren entstehen wenn man als Phasendifferenz einen anderen Wert als $\frac{\lambda}{4}$ nimmt. Bei 0 Grad sieht man außerdem eine normale Intensitätsverteilung da eine Amplitude des
  aufgespaltenen E Vecktors 0 wird (siehe GL 5). Wenn das Plättchen richtig gewesen wäre müsste man bei 45 Grad einen Kreis sehen.

\end{document}

