%        File: photo.tex
%     Created: Fre Dez 10 10:00  2010 C
% Last Change: Fre Dez 10 10:00  2010 C
%
\documentclass[a4paper]{article}
\usepackage[]{amsmath}
\usepackage[utf8]{inputenc}
\usepackage{ngerman}
\usepackage{graphicx}
\usepackage{fancyhdr}
\usepackage{lastpage}

\pagestyle{fancy}
\fancyhead{}
\fancyfoot{}
\fancyhead[L]{Armin Fürst \linebreak Michael Pusterhofer}
\fancyhead[C]{Photospektrometer}
\fancyhead[R]{G27}
\fancyfoot[C]{\thepage/\pageref{LastPage} }


\begin{document}

\section{Aufgabenstellung}
\begin{itemize}
  \item Messen der Intensitäten 2 verschiedenen Filtern
  \item Zeigen der Additivität der Absorbance
\end{itemize}

\section{Voraussetzungen}
\subsection{Transmission}
\begin{align}
				t=\frac{I}{I_0} \quad  \Delta t = \left|\frac{1}{I_0}\right|\Delta I + \left| \frac{I}{I_0^2} \right| \Delta I_0 \\
				\Delta I\dots \text{Fehler der Intensität mit Filter}\\
				I\dots \text{Intensität mit Filter}\\
				\Delta I_0\dots \text{Fehler der Intensität ohne Filter}\\
				I_0\dots \text{Intensität ohne Filter}
				\label{trans}
\end{align}

\subsection{Absorbance}
\begin{align}
				ad=Log\left( \frac{I_0}{I} \right) \quad
        \Delta ad = \left| \frac{1}{I_0}\right| \Delta I_0 + \left| \frac{1}{I}\right|\Delta I\\
				\Delta I\dots \text{Fehler der Intensität mit Filter}\\
				I\dots \text{Intensität mit Filter}\\
				\Delta I_0\dots \text{Fehler der Intensität ohne Filter}\\
				I_0\dots \text{Intensität ohne Filter}
				\label{adsorb}
\end{align}


\section{Versuchsaufbau}
Das Licht der Halogenlampe wird durch die Filter und dann durch den Monochromator geleitet, auf dessen Ausgang die Photodiode sitzt. 

\section{Geräteliste}
\begin{itemize}
  \item Magenter Filter
  \item Gelber Filter
  \item Voltmeter VC230
  \item Halogen-Lampe
  \item Photodiode
  \item Monochromator
\end{itemize}
\newpage

\section{Messung}

Diodenspannung ohne Licht
\begin{align}
  U_{D0}=(29\pm3)mV
  \label{diode}
\end{align}

\begin{table}[ht]
				\centering
				\caption{Intensitätswerte mit und ohne Filter}
				\begin{itemize}
								\item $\lambda$ \dots Wellenlänge in nm , U=$\pm$0.2nm
								\item $U_0$ \dots Intensität ohne Filter,U=$\pm$0.002V
								\item $U_{mag}$ \dots Intensität mit magentem Filter,U=$\pm$0.002V
								\item $U_{gelb}$ \dots Intensität mit gelbem Filter,U=$\pm$0.002V
								\item $U_{m+p}$ \dots Intensität mit gelbem und magentem Filter,U=$\pm$0.002V
				\end{itemize}
				\begin{tabular}{|c|c|c|c|c|}
								\hline
							$	\lambda/nm$  & $U_0 / V$ & $U_{mag}/V$ & $U_{gelb}/V$ & $U_{m+g}$ \\ \hline
 400 & 0.184 & 0.117 & 0.045 & 0.041 \\ \hline
 410 & 0.246 & 0.150 & 0.048 & 0.042 \\ \hline
 420 & 0.326 & 0.179 & 0.045 & 0.041 \\ \hline
 430 & 0.418 & 0.201 & 0.046 & 0.042 \\ \hline
 440 & 0.547 & 0.217 & 0.046 & 0.042 \\ \hline
 450 & 0.698 & 0.218 & 0.046 & 0.041 \\ \hline
 460 & 0.906 & 0.204 & 0.047 & 0.041 \\ \hline
 470 & 1.117 & 0.163 & 0.048 & 0.042 \\ \hline
 480 & 1.326 & 0.131 & 0.089 & 0.046 \\ \hline
 490 & 1.546 & 0.098 & 0.314 & 0.051 \\ \hline
 500 & 1.883 & 0.074 & 0.709 & 0.053 \\ \hline
 510 & 2.310 & 0.064 & 1.233 & 0.053 \\ \hline
 520 & 2.709 & 0.057 & 1.827 & 0.052 \\ \hline
 530 & 3.084 & 0.051 & 2.302 & 0.048 \\ \hline
 540 & 3.438 & 0.048 & 2.668 & 0.047 \\ \hline
 550 & 3.784 & 0.049 & 2.945 & 0.047 \\ \hline
 560 & 4.12 & 0.054 & 3.356 & 0.051 \\ \hline
 570 & 4.47 & 0.054 & 3.764 & 0.052 \\ \hline
 580 & 4.80 & 0.058 & 3.801 & 0.054 \\ \hline
 590 & 5.13 & 0.088 & 4.29 & 0.077 \\ \hline
 600 & 5.45 & 0.308 & 4.47 & 0.237\\\hline
				\end{tabular}
				\label{tab:1}
\end{table}
\newpage
\section{Auswertung}
\subsection{Transmission}
\begin{table}[ht]
				\centering
				\caption{Gerechnete Transmission}
        \begin{itemize}
          \item $t_{mag}$ \dots Transmission mit magentem Filter
          \item $\Delta t_{mag}$ \dots Größtfehler der Transmission mit magentem Filter
          \item $t_{gelb}$ \dots Transmission mit gelbem Filter 
          \item $\Delta t_{gelb}$ \dots Größtfehler der Transmission mit gelbem Filter
          \item $t_{m+g}$ \dots Transmission mit beiden Filtern 
          \item $\Delta t_{m+g}$ \dots Größtfehler der Transmission mit beiden Filtern
        \end{itemize}
				\begin{tabular}{|c|c|c|c|c|c|}
\hline
$t_{mag}$ & $\Delta t_{mag}$ & $t_{gelb}$ & $\Delta t_{gelb}$ & $t_{m+g}$ & $\Delta t_{m+g}$ \\
\hline
0.57 & 0.02 &    			0.10& 0.01&    				0.08& 0.01\\\hline
0.56 & 0.01 &    			0.09& 0.01&   				0.06& 0.01\\\hline
0.51 & 0.01 &   			0.054& 0.007&  				0.040& 0.010\\\hline
0.442 & 0.007 & 			0.044& 0.005&  				0.033& 0.005\\\hline
0.363 & 0.005 &  			0.033& 0.004&  				0.025& 0.004\\\hline
0.283 & 0.003 &  			0.025& 0.003&   			0.018& 0.003\\\hline
0.200 & 0.003 &  			0.021& 0.002&  				0.014& 0.002\\\hline
0.123 & 0.002 &  			0.017& 0.002&  				0.012& 0.002\\\hline
0.079 & 0.002 & 			0.046& 0.002&  				0.013& 0.002\\\hline
0.045 & 0.001 &  			0.188& 0.002&   			0.015& 0.001\\\hline
0.024 & 0.001 &  			0.367& 0.001&   			0.013& 0.001\\\hline
0.0153 & 0.0009 &			0.528& 0.001&   			0.0105& 0.0009\\\hline
0.0104 & 0.0008 &			0.671& 0.001&   			0.0086& 0.0008\\\hline
0.0072 & 0.0007 &			0.744& 0.001&   			0.0062& 0.0007\\\hline
0.0056 & 0.0006 &			0.774& 0.001&   			0.0053& 0.0006\\\hline
0.0053 & 0.0005 &			0.7766& 0.0009&  			0.0048& 0.0005\\\hline
0.0061 & 0.0005 &			0.8132& 0.0009&  			0.0054& 0.0005\\\hline
0.0056 & 0.0005 &			0.8410& 0.0008&  			0.0052& 0.0005\\\hline
0.0061 & 0.0004 &			0.7906& 0.0008&   		0.0052& 0.0004\\\hline
0.0116 & 0.0004 &			0.8353& 0.0007&  			0.0094& 0.0004\\\hline
0.0515 & 0.0004 &			0.8192& 0.0007&       0.0384& 0.0004\\\hline
				\end{tabular}
				\label{tab:trans}
\end{table}
\newpage
\subsection{Absorbance}
\begin{table}[ht]
  \centering
  \caption{Gerechnete Absorbance}
  \begin{itemize}
    \item $ad_{mag}$ \dots Absorbance des magenten Filters
    \item $\Delta ad_{mag}$ \dots Größtfehler der Absorbance des magenten Filters
    \item $ad_{gelb}$\dots Absorbance des gelben Filters
    \item $\Delta ad_{gelb}$\dots Größtfehler der Absorbance des gelben Filters
    \item $ad_{m+g}$\dots Absorbance beider Filter
    \item $\Delta ad_{m+g}$\dots Größtfehler der Absorbance beider Filter
  \end{itemize}
  \begin{tabular}{|c|c|c|c|c|c|}
    \hline
    $ad_{mag}$&$\Delta ad_{mag}$& $ad_{gelb}$&$\Delta ad_{gelb}$& $ad_{m+g}$&$\Delta ad_{m+g}$\\\hline
 0.566088 & 0.1&    6.81287 & 0.3&    2.55852 & 0.6\\\hline
 0.584107 & 0.08&   7.30676 & 0.2&    2.81495 & 0.5\\\hline
 0.683097 & 0.07&   8.76379 & 0.2&    3.20883 & 0.5\\\hline
 0.816085 & 0.06&   9.39147 & 0.2&    3.39863 & 0.5\\\hline
 1.01353 & 0.06&    10.2507 & 0.2&    3.68503 & 0.5\\\hline
 1.26404 & 0.05&    11.0181 & 0.2&    4.02088 & 0.5\\\hline
 1.61172 & 0.06&    11.6588 & 0.2&    4.2916 & 0.5\\\hline
 2.09426 & 0.07&    12.1434 & 0.2&    4.42715 & 0.5\\\hline
 2.54284 & 0.08&    9.22111 & 0.08&   4.3346 & 0.4\\\hline
 3.09038 & 0.1&     5.01851 & 0.04&   4.23345 & 0.3\\\hline
 3.71844 & 0.2&     3.0147 & 0.04&    4.34705 & 0.3\\\hline
 4.17702 & 0.2&     1.92676 & 0.04&   4.55432 & 0.3\\\hline
 4.56137 & 0.2&     1.21204 & 0.04&   4.75808 & 0.3\\\hline
 4.93349 & 0.3&     0.905453 & 0.05&  5.0801 & 0.3\\\hline
 5.18974 & 0.3&     0.789401 & 0.05&  5.2438 & 0.4\\\hline
 5.23511 & 0.3&     0.782186 & 0.05&  5.34047 & 0.4\\\hline
 5.09767 & 0.3&     0.647003 & 0.07&  5.2255 & 0.3\\\hline
 5.17976 & 0.3&     0.549507 & 0.08&  5.26314 & 0.3\\\hline
 5.10302 & 0.2&     0.73526 & 0.07&   5.25144 & 0.3\\\hline
 4.45965 & 0.1&     0.574118 & 0.08&  4.66599 & 0.2\\\hline
 2.96682 & 0.04&     0.633962 & 0.08& 3.2605 & 0.05\\  \hline
  \end{tabular}
  \label{tab:adsorb}
\end{table}

\newpage
\subsection{Diagramme}

\begin{figure}[ht]
  \begin{center}
    \includegraphics[width=0.7\textwidth]{trans.eps}
  \end{center}
  \caption{Transmission des Magenten Filters(magenta), des gelben Filters(gelb), beider Filter(rot) und das Produkt des magenten und gelben Filters(schwarz) über die Wellenlänge($\lambda$)}
  \label{fig:trans}
\end{figure}


\begin{figure}[ht]
  \begin{center}
    \includegraphics[width=0.7\textwidth]{adsorb.eps}
  \end{center}
  \caption{Absorbance des Magenten Filters(magenta), des gelben Filters(gelb), beider Filter(rot) und das Produkt des magenten und gelben Filters(schwarz) über die Wellenlänge($\lambda$) }
  \label{fig:adsob}
\end{figure}

\section{Zusammenfassung}
Bei größeren Wellenlängen sieht man gut, dass die Additivität der Absorbance gegeben ist. Die Spitze in der Kurve der Absorbance des gelben Filters ist wahrscheinlich auf die geringe Intensität zurückzuführen.
\end{document}

